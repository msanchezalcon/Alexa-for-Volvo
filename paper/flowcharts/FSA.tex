\begin{figure}[H]
    \centering
% Style
\tikzstyle{startstop} = [rectangle, rounded corners, minimum width=3cm, minimum height=1cm,text centered, draw=black, text width=3cm,fill=blue!30]
\tikzstyle{io} = [trapezium, trapezium left angle=70, trapezium right angle=110, minimum width=3cm, minimum height=1cm, text centered, draw=black, fill=blue!30]
\tikzstyle{process} = [rectangle, minimum width=3cm, minimum height=1cm, text centered, text width=3cm, draw=black]
\tikzstyle{decision} = [diamond, minimum width=3cm, minimum height=1cm, text centered, draw=black]

\tikzstyle{arrow} = [thick,->,>=stealth]

%Flowchart---------------------------------------------------------
\begin{tikzpicture}[node distance=2cm,on grid]
\node[state, initial, accepting]    (q1) {$invocation$};
\node[state, right of=q1]           (q2) {$state2$};
\node[state, right of=q2]           (q3) {$state3$};
\node[state, right of=q3]           (q4) {$state4$};
\node[state,accepting, right of=q4] (q5) {$fulfilled$};
\draw [arrow](q1) edge node [above] {\tt } (q2);
\draw [arrow](q2) edge node [above] {\tt } (q3);
\draw [arrow](q3) edge node [above] {\tt } (q4);
\draw [arrow](q4) edge node [above] {\tt } (q5);
\end{tikzpicture}
\caption{the transition states of a finite-state dialogue manager}
% give it a name
\label{fig:FSA}
\end{figure}